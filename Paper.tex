\documentclass[journal]{IEEEtran}

\usepackage{cite}

% *** GRAPHICS RELATED PACKAGES ***
%
\ifCLASSINFOpdf
  % \usepackage[pdftex]{graphicx}
  % declare the path(s) where your graphic files are
  % \graphicspath{{../pdf/}{../jpeg/}}
  % and their extensions so you won't have to specify these with
  % every instance of \includegraphics
  % \DeclareGraphicsExtensions{.pdf,.jpeg,.png}
\else
\fi



\begin{document}


\title{Comparación de técnicas basadas en datos para la predicción de potencia en una turbina a partir de datos meteorológicos de una estación cercana}


\author{Diego Lera, Antonio Martínez-Gandía}% <-this % stops a space




\markboth{Journal of \LaTeX\ Class Files,~Vol.~14, No.~8, August~2015}%
{Shell \MakeLowercase{\textit{et al.}}: Bare Demo of IEEEtran.cls for IEEE Journals}

\maketitle

% As a general rule, do not put math, special symbols or citations
% in the abstract or keywords.
\begin{abstract}
En este documento se presentan los resultados de la comparación de aplicar regresión logística, KNN, MARS y redes LSTM a las predicciones meteorológicas de la ciudad más cercana a una turbina eólica para calcular la velocidad del viento en ella y así conseguir a su vez una predicción de potencia
\end{abstract}

% Note that keywords are not normally used for peerreview papers.
\begin{IEEEkeywords}
KNN, MARS, LSTM, Regresion, paper, I+D+i.
\end{IEEEkeywords}







\IEEEpeerreviewmaketitle



\section{Introducción}

\IEEEPARstart{S}{i} queremos predecir la potencia generada por una planta de energía renovable necesitamos conocer las condiciones meteorológicas en los próximos días. Esta información sólo suele recopilarse en las estaciones meteorológicas, y éstas pueden encontrarse lejos de la planta. 

Como solución a este problema planteamos el uso de cuatro técnicas de \emph{machine learning}: Regresión Lineal, Regresión por K-vecinos (KNN), MARS y redes LSTM, que compararemos para descubrir cúal presenta mejores resultados.

% Da la sensación de que planteamos la solución antes que el problema, ¿no?
El problema en concreto se centra en  predecir la potencia de una turbina eólica situada en la provincia de Yalova, Turquía ( X:668478 Y:4494833 UTM ED 50, 6 degree), cuya estación meteorológica más cercana se encuentra en la capital, a 15 kilómetros de las turbinas.
Para el entrenamiento de los algoritmos utilizaremos dos bases de datos. Una con la relación entre la potencia generada por la turbina y las características del viento (velocidad y dirección) en su ubicación, y otra con los datos meteorológicos de la estación.

Estado del arte en cuanto a:
 \begin{itemize}
\item Predicción meteorológica con machine learning
Existen múltiples estudios destinados a la predicción meteorológica con técnicas de \emph{machine learning}, concretando para nuestro problema, el artículo de Xingjian Shi, Zhourong Chen, Hao Wang y Dit-Yan Yeung  para la predicción de precipitaciones usando redes LSTM\cite{NIPS2015_5955} arroja buenos resultados con una correlación de hasta un 0.908. Un estudio de la predicción de la irradiancia de un año entero realizado por Xiangyun Qing y Yugang Niu muestra mejoras respecto a otra red convolucional como BPNN \cite{QING2018461}
\item Predicción de potencia con machine learning

\item Combinado de potencia y meteorología
\item Explicar un poco cada método
\end{itemize}


 
% (should never be an issue)


\section{Métodos}
Para afrontar el problema lo dividimos en dos partes. En primer lugar, nos centramos en la predicción de la potencia con los datos del viento en la ubicación de la turbina. En segundo lugar calcularemos el viento en dicha ubicación a partir de los datos de la estación meteorológica de Yalova. 

Los datos han de estar tomados en los mismos instantes de tiempo. El \emph{dataset} de la potencia de la turbina nos ofrece los datos tomados cada diez minutos, comenzando el 01/01/2018 y terminando el 31/12/2018.  Por otro lado, la estación meteorológica nos ofrece un histórico de los datos del mismo año con una periodicidad de 3 horas, por lo que, para la predicción de viento, debemos eliminar los datos no coincidentes.  

Para realizar una limpieza de los datos de la turbina, se realiza una eliminación de los datos que correspondan a: paros en la turbina ($P=0$) y momentos en los que la potencia teórica y real difieren en exceso ($P_{teorica}>P_{real}+500$).

Preparación de los datos:
Dividerand
Preparadatos LSTM, lapso.
% An example of a floating figure using the graphicx package.
% Note that \label must occur AFTER (or within) \caption.
% For figures, \caption should occur after the \includegraphics.
% Note that IEEEtran v1.7 and later has special internal code that
% is designed to preserve the operation of \label within \caption
% even when the captionsoff option is in effect. However, because
% of issues like this, it may be the safest practice to put all your
% \label just after \caption rather than within \caption{}.
%
% Reminder: the "draftcls" or "draftclsnofoot", not "draft", class
% option should be used if it is desired that the figures are to be
% displayed while in draft mode.
%
%\begin{figure}[!t]
%\centering
%\includegraphics[width=2.5in]{myfigure}
% where an .eps filename suffix will be assumed under latex, 
% and a .pdf suffix will be assumed for pdflatex; or what has been declared
% via \DeclareGraphicsExtensions.
%\caption{Simulation results for the network.}
%\label{fig_sim}
%\end{figure}

% Note that the IEEE typically puts floats only at the top, even when this
% results in a large percentage of a column being occupied by floats.


% An example of a double column floating figure using two subfigures.
% (The subfig.sty package must be loaded for this to work.)
% The subfigure \label commands are set within each subfloat command,
% and the \label for the overall figure must come after \caption.
% \hfil is used as a separator to get equal spacing.
% Watch out that the combined width of all the subfigures on a 
% line do not exceed the text width or a line break will occur.
%
%\begin{figure*}[!t]
%\centering
%\subfloat[Case I]{\includegraphics[width=2.5in]{box}%
%\label{fig_first_case}}
%\hfil
%\subfloat[Case II]{\includegraphics[width=2.5in]{box}%
%\label{fig_second_case}}
%\caption{Simulation results for the network.}
%\label{fig_sim}
%\end{figure*}
%
% Note that often IEEE papers with subfigures do not employ subfigure
% captions (using the optional argument to \subfloat[]), but instead will
% reference/describe all of them (a), (b), etc., within the main caption.
% Be aware that for subfig.sty to generate the (a), (b), etc., subfigure
% labels, the optional argument to \subfloat must be present. If a
% subcaption is not desired, just leave its contents blank,
% e.g., \subfloat[].


% An example of a floating table. Note that, for IEEE style tables, the
% \caption command should come BEFORE the table and, given that table
% captions serve much like titles, are usually capitalized except for words
% such as a, an, and, as, at, but, by, for, in, nor, of, on, or, the, to
% and up, which are usually not capitalized unless they are the first or
% last word of the caption. Table text will default to \footnotesize as
% the IEEE normally uses this smaller font for tables.
% The \label must come after \caption as always.
%
%\begin{table}[!t]
%% increase table row spacing, adjust to taste
%\renewcommand{\arraystretch}{1.3}
% if using array.sty, it might be a good idea to tweak the value of
% \extrarowheight as needed to properly center the text within the cells
%\caption{An Example of a Table}
%\label{table_example}
%\centering
%% Some packages, such as MDW tools, offer better commands for making tables
%% than the plain LaTeX2e tabular which is used here.
%\begin{tabular}{|c||c|}
%\hline
%One & Two\\
%\hline
%Three & Four\\
%\hline
%\end{tabular}
%\end{table}


\section{Conclusion}
The conclusion goes here.





% if have a single appendix:
%\appendix[Proof of the Zonklar Equations]
% or
%\appendix  % for no appendix heading
% do not use \section anymore after \appendix, only \section*
% is possibly needed

% use appendices with more than one appendix
% then use \section to start each appendix
% you must declare a \section before using any
% \subsection or using \label (\appendices by itself
% starts a section numbered zero.)
%


\appendices
\section{Proof of the First Zonklar Equation}
Appendix one text goes here.

% you can choose not to have a title for an appendix
% if you want by leaving the argument blank
\section{}
Appendix two text goes here.


% use section* for acknowledgment
\section*{Acknowledgment}


The authors would like to thank...


% Can use something like this to put references on a page
% by themselves when using endfloat and the captionsoff option.
\ifCLASSOPTIONcaptionsoff
  \newpage
\fi



% trigger a \newpage just before the given reference
% number - used to balance the columns on the last page
% adjust value as needed - may need to be readjusted if
% the document is modified later
%\IEEEtriggeratref{8}
% The "triggered" command can be changed if desired:
%\IEEEtriggercmd{\enlargethispage{-5in}}

% references section

% can use a bibliography generated by BibTeX as a .bbl file
% BibTeX documentation can be easily obtained at:
% http://mirror.ctan.org/biblio/bibtex/contrib/doc/
% The IEEEtran BibTeX style support page is at:
% http://www.michaelshell.org/tex/ieeetran/bibtex/
%\bibliographystyle{IEEEtran}
% argument is your BibTeX string definitions and bibliography database(s)
%\bibliography{IEEEabrv,../bib/paper}
%
% <OR> manually copy in the resultant .bbl file
% set second argument of \begin to the number of references
% (used to reserve space for the reference number labels box)
% \begin{thebibliography}{1}

% \bibitem{IEEEhowto:kopka}
% H.~Kopka and P.~W. Daly, \emph{A Guide to \LaTeX}, 3rd~ed.\hskip 1em plus
%   0.5em minus 0.4em\relax Harlow, England: Addison-Wesley, 1999.

% \bibliography{convolutional-lstm-network-a-machine-learning-approach-for-precipitation-nowcasting}{}       % expects file "myrefs.bib"

% \end{thebibliography}

% biography section
% 
% If you have an EPS/PDF photo (graphicx package needed) extra braces are
% needed around the contents of the optional argument to biography to prevent
% the LaTeX parser from getting confused when it sees the complicated
% \includegraphics command within an optional argument. (You could create
% your own custom macro containing the \includegraphics command to make things
% simpler here.)
%\begin{IEEEbiography}[{\includegraphics[width=1in,height=1.25in,clip,keepaspectratio]{mshell}}]{Michael Shell}
% or if you just want to reserve a space for a photo:


% You can push biographies down or up by placing
% a \vfill before or after them. The appropriate
% use of \vfill depends on what kind of text is
% on the last page and whether or not the columns
% are being equalized.

%\vfill

% Can be used to pull up biographies so that the bottom of the last one
% is flush with the other column.
%\enlargethispage{-5in}

\bibliography{Bibliografia}

\bibliographystyle{IEEEtran}

% that's all folks
\end{document}
